\section{Conclusion}
\label{sec:conc}

In this work, we first presented a measurement study of the power behavior
of OLED displays on modern smartphones.
Our study shows they violate both the
superposition property and the monotonicity property and explains the
accuracy limitations of prior OLED power models. We then presented a
novel OLED power model based on a key
insight that the OLED power draw is relatively smooth within small
subgrids of the large 3-D sRGB color space, and hence each can be fitted
well via linear regressions on the linear subpixel power functions.
Our new OLED power model reduces the prediction error of prior models
from 12.8\% to 3.0\% on average across 3 recent generations of Android
phones.
Using the new OLED power model, we developed two OLED power
profiling tools: a per-frame OLED power profiler and an enhanced Android Battery.
% that accurately accounts for per-app screen energy.
%
Further, we performed to our knowledge the first power saving
measurement of the emerging dark mode for a set of 6 Google
apps to showcase how the two tools can help developers
and phone users to understand and optimize the battery drain of
different app GUI designs.
%We plan to open source both tools to the Android developers community.
The \appwithlink app
% \href{https://play.google.com/store/apps/details?id=com.pdeveloper.pcav5}{\name app}
has been released in Google Play, and
we
% have the \name app in Google Play and
plan to open source \name in \url{github} and Battery+ into AOSP
to facilitate further research and development.



\if 0
Our study shows that at brightness level 50\%
switching to dark mode reduces the OLED
power draw on average by 45\%, 24\% and 21\% on Nexus 6,
Pixel 2 and Moto Z3 for the set of
Google apps which translates into 13\%, 6\% and 7\%
average total phone power draw reduction on the 3 phones, respectively.
\fi
