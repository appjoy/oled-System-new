\begin{abstract}

By omitting external lighting, OLED display significantly reduces the
power draw compared to its predecessor LCD and has gained wide
adoption in modern smartphones.  The real potential of OLED in saving
phone battery drain lies in exploiting app UI color design, \ie how to
design app UI to use pixel colors that result in low OLED display
power draw.
%   Doing so, however, requires a per-frame OLED power
%   profiler than can accurately estimate the OLED power draw of each
%   displayed frame on a given device and its breakdown among the various
%   constituting UI components.
In this paper, we design and implement
an accurate per-frame OLED display power profiler, \namee, that helps
developers to gain insight into the impact of different app UI design
on its OLED power draw, and an enhanced Android Battery that
helps phone users to understand and manage phone
display energy drain, for example, from different app and display
configurations such as dark mode and screen brightness. A major challenge in
designing both tools is to develop an accurate and robust OLED display
power model.  We experimentally show that the linear-regression-based
OLED power models developed in the past decade cannot capture the
unique behavior of OLED display hardware in modern smartphones which have
a large color space and propose a new piecewise power model that 
achieves much better modeling accuracy than the prior-art
by applying linear regression in each small regions of the vast color
space.
Using the two tools, we performed to our knowledge the first power
saving measurement of the emerging dark mode for a set of 6 Google
Android apps.

%  We demonstrate how the \name profiler can be used by app developers to
%  easily quantify the impact of dark mode on the OLED
%  power draw  and further gain insight into the OED power breakdowns
%  at the UI-component granularity.

\if 0
Accurate OLED power modeling is essential in power management such
as energy profiling and optimization of mobile apps and devices.
Since OLED power draw depends on the colors of the pixels displayed,
accurate modeling of OLED displays is challenging as modern phone OLED
displays come with millions of pixels, each with a huge color space
($256^3$).  In this paper, we first present a measurement study to
characterize the power behavior of OLED displays on modern
smartphones, which illustrates the limitations of the prior OLED power
models. We then present a novel OLED power model that overcomes the
challenges presented by the unique OLED display power behavior on
modern phones. We show the new model reduces the OLED power prediction error
of the prior art from
\nexuserror, \pixelerror, and \motoerror,
% 5.8\%, 23.8\%, and 13.4\%
to 3.3\%, 3.3\%, and 2.9\% across 3 recent generations of Android phones.
Third, we develop a per-frame OLED power profiler that accurately breaks down the
OLED power in displaying a frame among the UI components.
Finally, using the OLED power profiler, we quantify
the OLED and whole phone power savings for a set of 9 Google and Android system apps that
recently rolled out with support for cm``dark mode'' -- the latest power saving
scheme embraced by
% dominating mobile OS players such as
both Android and
iOS.  Our study shows that at brightness level 50\%
switching to dark mode reduces the OLED
power draw on average by 45\%, 22\% and 21\% on Nexus 6,
Pixel 2 and Moto Z3 for the set of
Google apps which translates into 13\%, 6\% and 7\%
average total phone power draw reduction on the 3 phones, respectively.
The power savings are significantly higher at 100\% brightness level.
\fi

\end{abstract}
