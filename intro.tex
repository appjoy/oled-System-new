\vspace{-0.01in}

\section{Introduction}
\label{sec:intro}

Optimizing the battery drain of mobile apps helps to extend the
mobile device battery life which is critical to the mobile experience
of the 5+ billion phone users (over half are smartphones)~\cite{gsma2019}.
Optimizing the battery drain of mobile
apps requires optimizing the battery drain of all power-hungry
device components, including CPU, GPU, display, WiFi/LTE,
GPS, and hardware decoder. After over a decade of phone evolution, display
has remained as one of the major power-draining components in modern
smartphones~\cite{dong2009current,dong:2011chameleon,chen:2013:display,wan:2017:displayenergy,chen:2016:dac,chan:2016:image,crayon:eurosys16}.

% The display technology has also evolved over the years.
The latest display technology, Organic light-emitting diode (OLED)~\cite{oled:2003,oled:2004},
uses light-emitting diode as pixels and therefore omits the need for
external backlight used in its predecessor LCD, and in doing so
provides much better power efficiency than LCD.  Due to its power
efficiency and several other advantages (thinner, lighter, and more
flexible) over LCD and LED, OLED has seen wide adoption in high-end and mid-range
smartphones.

Since the OLED power draw is directly related to the displayed
content, the real potential of OLED in terms of power savings lies in
exploiting the app UI color design, \ie how to design the app UI 
to use pixel colors that result in less OLED display power draw.
These include manual designs by app developers as well as
automatic color transformations~\cite{dong:ispled09,dong:2011chameleon,crayon:eurosys16}.
One of the extreme examples of color transformation to save display energy of mobile
devices is the recent wide adoption of 
dark-themed color contents, known as dark mode,
% \footnote{An old idea first supported in modern OS as early as 1991 (Apple's System 7 OS).}
% which are known to draw much less power than white-theme color contents. 
%  the mobile industry did not catch on until late 2018
by both Android and iOS which added dark mode as one of major
features in their 2019 OS
update~\cite{googledevsummit2018,appleaccouncedarkmode}, and app
vendors who quickly rolled out dark mode UI options in their latest
app releases~\cite{darkModeActicle1,darkModeActicle2,darkModeActicle3,darkModeActicle4,darkModeActicle5,darkModeActicle6,darkModeActicle7}.

Despite the industry's major push for dark mode, it remains unclear
how much power and energy savings dark mode will bring to the apps, as
the OLED display power saving from switching from the normal (or
light) to dark mode depends on the displayed content, which can vary
significantly from one app to another, and from one app activity to
the next. More generally, the industry is lacking an OLED display
power profiling tool that can accurately estimate the OLED display
power draw and attribute it to the individual UI components.  Such a
tool will enable an app developer to gain instant insight into how different
UI designs affect the OLED display power draw in running the app on
specific mobile devices and optimize the app UI design accordingly.

\if 0
In this paper, we undertake to our knowledge the first measurement
study of the energy savings of dark mode for a set of Google/Android system
apps that recently rolled out support for dark mode.
% ,as well as the upcoming Android Q (using its beta release) dark theme.
\fi

In this paper, we present the design and implementation
of two OLED power management tools:
an accurate per-frame OLED display power profiler, \namee, that helps
developers to gain insight into the impact of different app UI designs
on its OLED power draw, and an enhanced Android Battery called Android Battery+ 
(or Battery+ for short) that
helps phone users to understand and manage phone
display energy drain, for example, from different app and display
configurations such as dark mode and screen brightness. A major challenge in
designing both tools lies
in designing an accurate and robust
OLED power model that accurately estimates the display power draw
as a function of the displayed content on any device.
%   , as either the
%   external power meter or the internal power sensor can only measure the
%   total phone power draw by all its active components; (2) We need to
%   identify all the UI components of a displayed frame, \eg separating
%   dynamic objects from static objects, in order to calculate 
%   the OLED power draw by each UI component.

\if 0
An accurate OLED power model enables app developers to
% isolate and
quantify the OLED display power/energy savings which has many
applications, such as studying the power savings from switching to
dark mode for any apps running in the lab or ``in the wild'' on real
users' phones
% \footnote{We have developed an app that records the
% screen and accurately caculates the OLED energy drain throughout the   day.}
and more generally energy profiling of mobile
apps~\cite{appscope,zhang2010accurate,shye2009into,pathak:eurosys12,mittal:mobicom12,androidprofiler}
which performs component-wise energy accounting at the app source-code
level.
\fi


Since power modeling of OLED display has been studied
over the past decade, first on external display then
earlier smartphones~\cite{dong2009current,kim2013runtime,park2015accurate},
we first study {\em if the prior-art power models of OLED display proposed
  in the past decade can accurately capture the power behavior of OLED
  displays on modern smartphones.}  To answer this question, we
conduct a measurement study to characterize the power behavior of OLED
displays on three representative smartphones from 
three recent generations, Pixel 2 (2017), Moto Z3 (2018), and Pixel 4 (2019).
%
Our measurement study reveals
unique power draw behavior of modern phone OLED displays: (1) The OLED
display power draw violates the superposition property observed
in~\cite{dong2009current}, \eg the power draw in displaying
the white color can be far less than the simple summation of those in
displaying the three base colors: red, green and blue (after removing
base power).  (2) The OLED
display power often violates the  monotonicity principle, that
the power draw in displaying canonically larger RGB values in the color
space should be higher.
\forjnl{(3) New features in
recent phones such as color mode, color correction
as well as adjusting the hardware brightness button
directly affect the OLED display power draw
via color transformation in hardware composer,
which does not change the pixel
values in the frame buffer (what is recordable from the OS-provided
API).
}
The two findings suggest that previously proposed
linear and non-linear OLED power models are unlikely to achieve high
prediction accuracy, as confirmed by our experiments.
\forjnl{
  and the third suggests that any OLED power model based on frame buffer pixel values
needs to adjust for color transformations that can happen in the hardware composer.
}

Second, we study {\em how to develop new practical OLED power models that
  can accurately and robustly capture the power behavior of modern smartphone OLED
  displays.}  To develop such a power model,
we first unearth the root cause for why previous models fail:
%we    plot the OLED
%    power draw in displaying static chromatic images of pixels that span
%    the 3-D color space and found
% we first uncovered that the fundamental reason that
linear regression of subpixel power functions (which are linear in linear RGB
values)
% \footnote{Linear RGB values are RGB valeus with gamma-correction.}, respectively)
cannot capture the OLED power behavior of modern OLED displays
% is that
because any linear combinations of linear functions 
can only fit a flat plane, while the OLED power
values across the color space do not fall into a flat plane.

Our new OLED power model is based on a key observation that though the
pixel power values in the color space do not fall in a plane,
they form a reasonably smooth surface.  Therefore, if we cut
the 3-D color space into small subgrids, the pixel power values within
each subgrid are likely to fit closely in a flat plane and thus can be
fitted well using linear regressions of the (linear) subpixel power
functions. We experimentally show that our new piecewise OLED power model reduces the
OLED prediction error of the prior art~\cite{park2015accurate} by
5.1$\times$, 5.5$\times$ and 4.4$\times$ for dark
and 4.2$\times$, 2.3$\times$ and 2.5$\times$ for light mode for the three phones,
respectively, for a set of activities across 6 popular google apps.
%% NLM model vs P-NLM model

% 7.2$\times$-4.6$\times$,
% from
% \nexuserror, \pixelerror, and \motoerror,
% down to 3.3\%, 3.3\%, and 2.9\% on the three phones,
% respectively, for a set of 100 images with diverse graphical content.

%   The new OLED power model has many potential applications, such as
%   building fine-grained energy which requires accurate power model for
%   every power-hungry phone component.
%   profiler~\cite{appscope,zhang2010accurate,shye2009into,pathak:eurosys12,mittal:mobicom12,androidprofiler}.

Third, we design and implement the two tools on top of our new OLED
power model. \name accurately breaks down the OLED display power in
displaying a frame among its UI components and faces the challenge of
identifying all the UI components of a displayed frame, \eg an app
activity.
% , in order to calculate the OLED power draw by each UI component.
We design \name to use only standard Android services so
it is portable to any unmodified Android versions.
% {without superuser access}.  
Battery+ replaces
the simple screen energy accounting of Android Battery with an
accurate, per-app screen energy estimator.  Like Android
Battery, it runs in real time in the background and thus needs to have low
overhead. We carefully apply pixel sampling to both screen recording
and model application to control its average CPU utilization to be
only 1.1\%, 0.9\%, and 0.6\% on Pixel 2, Moto Z3, and Pixel 4, respectively,
while minimally affecting the power estimation accuracy.

%  to identify UI components
%  , that accurately breaks down the OLED power draw
% in displaying a frame among its constituting UI components.  We design
% \name to be easy-to-use -- it runs as an app,
% portable to different Android versions -- it only uses standard Android services
%  to identify UI components, and efficient by using pixel sampling -- analyzing a frame takes
% less than 8 seconds on all three phones.

%    , and an Enhanced
%    Android Battery that accurately measures the screen energy of all apps
%    running on the phone and assists phone users to adjust screen
%    brightness to elongate battery life.

Finally, 
%   how the OLED power profiler can be used by
%   developers and phone users to understand and optimize the battery drain
%   of different app UI designs,
we showcase the usage of the two tools by performing a case study of the dark mode
power saving that answers {\em how much does dark mode reduce the
  OLED display power draw and energy drain for real apps?}  In
particular, we study the power consumption of a set of 6
 Google apps {(each with 1B+ downloads)} in two ways.  First, {\em from the phone user's
  perspective,} we analyze the power draw of each Google app under a
typical user interaction using Battery+.
%   that lasts about 40 seconds which is driven
%   by UIAutomator 2~\cite{uianimator} and Appium~\cite{appium} to ensure
%   the same user interactions are carried out under light and dark modes.
Battery+ shows that (1) switching from light to dark mode
significantly reduces the average OLED power draw across all apps,
though the reduction varies significantly among the apps and the phones, ranging
between 4\%-34\% (average 24\%), 12\%-35\% (average 25\%), and
12\%-24\% (average 19\%) at the 50\% brightness level,
%    , and
%    48\%-67\% (average 58\%), 
%    64\%-81\% (average 72\%),
%    and 62\%-76\% (average 68\%) for 100\% brightness level,
on the 3 phones, respectively.  (2) The significant OLED power reduction
translates into moerate reduction in the total phone power draw, ranging between
-3\%-15\% (average 7\%),
-12\%-13\% (average 5\%),
and 0\%-15\% (average 10\%) at the 50\% brightness level
%  2\%-29\% (average 15\%),
%  16\%-46\% (average 32\%),
%  and 24\%-59\% (average 43\%)
across the apps on the 3 phones, respectively.

Second, {\em from the app developer's perspective}, we use \name to analyze how
much power the UI design change of each activity in dark mode saves in
the 6 Google apps.
%  We manually navigate each of
%  the apps to stop at all its major activities when running under 
%  dark and light mode, and capture the screen via screencap, and
%  then apply our new OLED power model to the captured screens to
%  calculate the OLED power draw.
\if 0
(1) Switching from light mode to dark mode significantly reduces the OLED power
draw of the app activities, ranging between 48\%--67\% (average 58\%),
64\%--81\% (average 72\%), and 62\%--76\% (average 68\%),
for brightness level 100\% across the apps on the 3 phones, respectively,
%  (2) The drastic OLED power saving in switching to dark mode 
which translates into significant saving of whole phone power draw,
on average 26\%, 39\%, and 47\% across the apps on the 3 phones, respectively.
(2) More importantly,
\fi
Our study shows how
\name quantifies and displays the power saving for each UI component
in switching to the new color scheme in dark mode, allowing app developers
to easily experiment with the design-power tradeoff of
alternative dark mode designs.

%  To our best knowledge, this is the first public measurement study of
%  OLED display power savings by dark mode on smartphones.

% Fourth, {\em can we develop design guidelines for app vendors who are thinking about developing dark mode and want to estimate the potential energy saving before investing the development effort?} 

We summarize the contributions of our study as follows:
\begin{itemize}[leftmargin=*]
  \item We experimentally characterize the unique power draw behavior
    of OLED displays on 3 representative modern smartphones and show
    how they violate the underlying assumptions made in prior OLED
    power models proposed in the past decade.
  \item We propose a new piecewise OLED power model and experimentally
    show that it reduces the prediction error of the prior art by 
%    \pixeltwoerrorreduction,
%    \motozthreeerrorreduction, and
%    \pixelfourerrorreduction
5.1$\times$, 5.5$\times$ and 4.4$\times$ for dark mode
and 4.2$\times$, 2.3$\times$ and 2.5$\times$ for light mode 
    on Pixel 2, Moto Z3, and Pixel 4, respectively.
  \item Using our new model, we develop a per-frame OLED
    power profiler that enables developers to instantly understand the
    OLED power draw breakdowns of activity UI designs to guide UI power optimization and an enhanced Android Battery that enables phone users to
understand and manage phone
display energy drain, for example, from different app and display
configurations such as dark mode and screen brightness.
%     , and an
%     Enhanced Android Battery that accurately measures the screen
%     energy of all apps running on the phone and assists phone users to
%     adjust screen brightness to elongate battery life.  
  \item Using the two tools, we present to our knowledge
    the first public study of the OLED power and phone power
    savings in switching from light mode to dark mode for a
    set of Google apps {(each with 1B+ downloads).}
  \end{itemize}




\if 0
Light-on-dark color scheme, or dark mode in short, refers to the color
scheme that uses light-colored text, icons, and graphical user
interface elements on a dark background.  While dark mode
is known to reduce the power on display technologies in general,
its saving on OLED is particularly pronounced, due to the unique power
characteristics of OLED displays.  For this reason, both Google and
Apple announced in late 2018 that dark mode would be made available on
Android and iOS devices. Since then an increasingly number of apps
(by toggling ``Enable Dark Theme'' in the app's settings) has rolled out
the dark mode option, and the Android system UI also added the dark
theme.

While it is generally understood that using dark color will lower the OLED
display power consumption, it is much less clear on how much
energy savings will dark mode bring to individual apps, as the
OLED display power saving in switching from light to 
dark mode depends on the displayed content and which may vary
significantly from one app to another. In this section, we conduct a
measurement study of the energy savings of dark mode for a set of
preinstalled Google apps which are hugely popular and thus will affect
a large number of mobile users by adopting dark mode.
The improved accuracy of the new model comes with negligible overhead:
64KB for storing 4K 4-coefficient linear regression models
and 678ms - 1100ms in applying the model to one complete frame on the 3 phones.

\fi
